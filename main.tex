% Made for Moscow Aviation Institute freely by (c)Liudmila Chizhikova


\documentclass[letterpaper,12pt]{article}
\usepackage[utf8]{inputenc}

%\usepackage[T2A]{fontenc}
\usepackage{mathptmx}
\usepackage{amsmath} %для отображения математических формул
\usepackage[russian,english]{babel}
\usepackage{hyphenat}


%\usepackage[T1]{fontenc}

\usepackage{array}
\usepackage{tabularx}
\usepackage{graphicx}
\usepackage{authblk}
\usepackage{fancyhdr}
\usepackage{url}
\usepackage{titlesec}
\usepackage{xcolor}
\usepackage{multicol}  
\usepackage[russian]{babel}

\usepackage[utf8]{inputenc}	
\usepackage[T2A]{fontenc}	
\usepackage{ifthen}	
\usepackage[english, russian]{babel}	
\usepackage{footmisc}
\usepackage{fancyhdr}
\usepackage{misccorr}
\usepackage{mathptmx}
\usepackage{textcomp}
\usepackage{multirow}
\usepackage{wrapfig}
\usepackage{epstopdf}
\usepackage{eso-pic}						%russian language
\usepackage[top=2.5cm, left=2.5cm, right=2.5cm, bottom=2.7cm]{geometry} %pagesize

\usepackage{type1ec}							%true type fonts cm
\usepackage{graphicx}							%for graphic \includegraphics
\usepackage{caption}                            %for cation separator
\usepackage{indentfirst}							%first abzac
\usepackage{amsfonts}							%math fonts
\usepackage{rotating}
\usepackage{mathtext}
\usepackage{amssymb}
\usepackage[T2A]{fontenc}
\usepackage[utf8]{inputenc}
\usepackage[russian]{babel}
\usepackage{floatrow,caption}
\usepackage[lofdepth]{subfig}
\usepackage{tempora}

\usepackage[
backend=biber,
style=alphabetic,
]{biblatex}



\begin{document}

\selectlanguage{russian}

%Made for Moscow Aviation Institute freely by (c)Liudmila Chizhikova

\begin{titlepage}

\begin{center}
\includegraphics[width=3cm, height=3cm]{mai_logo.png}    
\end{center}
\begin{center}
    \textbf{\Large Федеральное государственное бюджетное образовательное
учреждение высшего образования «Московский авиационный институт\newline
(национальный исследовательский университет)»  }
\end{center}
\vspace{30pt}
\begin{center}
    Работа по дисциплине :
    % введите название дисциплины в скобках
    \textbf{  }
\end{center}
\vspace{30pt}

%введите свое имя и имя преподавателя в файлах Student name.tex и Teacher name.tex
%Made for Moscow Aviation Institute freely by (c)Liudmila Chizhikova

\begin{flushright} %введите свое ФИО в скобках ниже
Работу выполнил \textit{ }
\end{flushright}

\begin{flushright} %введите свой курс ниже в пустых скобках
\textit{ }
\end{flushright}

\begin{flushright} %введите номер группы в пустых скобках ниже
\textit{ }
\end{flushright}
\vspace{20pt}

\endinput
%Made for Moscow Aviation Institute freely by (c)Liudmila Chizhikova

\begin{flushright} %введите должность преподавателя в пустых скобках ниже
Проверил \textit{ }
\end{flushright}

%Введите имя преподавателя в пустых скобках ниже
\begin{flushright}
\textit{ }
\end{flushright}

\endinput


\vspace{150pt}
\begin{center}
    Москва
    2021
\end{center}
\end{titlepage}
\endinput

\endinput

%Made for Moscow Aviation Institute freely by (c)Liudmila Chizhikova

\section{Предмет, тема, цель работы}
%введите текст здесь

\endinput
%Made for Moscow Aviation Institute freely by (c)Liudmila Chizhikova

\section{Метод или методология проведения работы}
%введите текст здесь

\endinput
%Made for Moscow Aviation Institute freely by (c)Liudmila Chizhikova

\section{Результаты работы}
%введите текст здесь

\endinput
%Made for Moscow Aviation Institute freely by (c)Liudmila Chizhikova

\section{Область применения результатов}
%введите текст здесь

\endinput
%Made for Moscow Aviation Institute freely by (c)Liudmila Chizhikova

\section{Выводы. Заключение}
%введите текст здесь

\endinput


%Библиография
\begin{thebibliography}{9}
\bibitem{texbook}
В.А. Иванов Математические основы теории оптимального и логического управления: учеб.пособие /В.А. Иванов , В.С. Медведев, Москва, Изд-во МГТУ им.Баумана, 2011 -599 стр. 

\bibitem{lamport94}


\end{thebibliography}


\end{document}
